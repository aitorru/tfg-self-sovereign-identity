\chapter{Introducción}\label{Introducción}
%Función que crea el título de capítulo y al cual se le da el nombre deseado a través de su parámetro obligatorio. Al no tener la función el “*” se escribirá también en el título del documento las palabras “Capítulo 1: …”. Además se indica, mediante la función “\label”, la correspondiente etiqueta que lleva asociada. La etiqueta sirve para que en caso de que luego se quiera hacer referencia al capítulo se haga llamando etiqueta tal que se escribiría “La información correspondiente a dicho tema se encuentra en el capítulo \ref{Int}.”

\thispagestyle{fancy}
%Función que determina que durante este capítulo se aplique el estilo Fancy.

\fancyhead[LE]{\thechapter.Introducción} 
%Función que se utiliza para indicar que en las páginas impares, aparezca en el encabezado en la parte izquierda, el número del capítulo con su correspondiente nombre.

En este capitulo, se van a introducir de manera breve las tecnologías que van a ser utilizadas en este proyecto. No se entrara en mucho detalle ya que todas las tecnologías que se van a referenciar, se van a ver de nuevo a lo largo del proyecto.
%Texto sin sentido y predeterminado que se irá utilizando a lo largo de todo el documento para simular la escritura de la memoria.
\section{SSI}
Self Sovereign identity, es una acercamiento de identidad digital. Dándole de vuelta la identidad a las personas. En la actualidad estamos muy acostumbrados a pantallas de login donde nos preguntan por un usuario y contraseña. Algunos pueden ser mas sofisticados que otros, pero todos se basan en el mismo concepto.
Con el SSI se intentaría eliminar la necesidad de introducir esos formularios, sustituyéndolos por lecturas de identidades propias. En vez de la pagina autorizarnos a nosotros, nosotros autorizamos a la pagina a saber quienes somos.

\section*{Descentralización}
IPFS y la blockchain, son una red de ordenadores. Estos, desbloquean las puertas de la descentralización, ya que permiten comunicar a personas a lo largo del mundo de manera segura y sin necesidad de infraestructuras caras ni grandes servidores.

\section*{Motivación}
En la web 2.0, las paginas basan su negocio en la compra venta de los datos de sus usuarios.

\newpage
\thispagestyle{empty}
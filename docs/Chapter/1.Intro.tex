\chapter{Introducción}\label{Introducción}
%Función que crea el título de capítulo y al cual se le da el nombre deseado a través de su parámetro obligatorio. Al no tener la función el “*” se escribirá también en el título del documento las palabras “Capítulo 1: …”. Además se indica, mediante la función “\label”, la correspondiente etiqueta que lleva asociada. La etiqueta sirve para que en caso de que luego se quiera hacer referencia al capítulo se haga llamando etiqueta tal que se escribiría “La información correspondiente a dicho tema se encuentra en el capítulo \ref{Int}.”

\thispagestyle{fancy}
%Función que determina que durante este capítulo se aplique el estilo Fancy.

\fancyhead[LE]{\thechapter.Introducción} 
%Función que se utiliza para indicar que en las páginas impares, aparezca en el encabezado en la parte izquierda, el número del capítulo con su correspondiente nombre.

En este capitulo se realizará una introducción al dominio en el que se desarrolla este proyecto de fin de grado y los conceptos esenciales en torno a los que ha girado el mismo. No se tratará con mucho detalle ya que todas las tecnologías que se van a referenciar se van a ver de nuevo a lo largo del proyecto. Una vez introducido el contexto del proyecto, se explicará la motivación para su realización.

\section{SSI}
\textit{Self-Sovering Identity} (SSI) o Identidad Autosoberana es la forma en que los titulares de la identidad individual crean sus credenciales y controlan cómo se comparten y usan sus datos personales \cite{web:ssi_wikipedia}.

La identidad digital era la transcripción a Internet de la identidad física de una persona, pero además incluye los datos que proporcionamos en la red: perfiles, fotos, correo electrónico, dirección, datos bancarios \dots,  y los que obtienen los sitios: navegación, preferencias, compras \dots

Identidad Digital 2.0 es un método de identificación en transacciones simple y abierto, similar a los procesos del mundo real \cite{web:identidad2o_wikipedia}. Es todo lo que manifestamos en el ciberespacio: nuestras actuaciones y la forma en la que nos perciben los demás en la red \cite{web:participacion}.

La identidad digital autogestionada es la propiedad soberana de los datos propios y credenciales recibidas y su gestión para compartirlos cómo y con quien se quiera y durante el tiempo deseado. Se comparte que se cumplen los requisitos de residencia o edad, o que se posee el título o la credencial necesarios. Además, por la autogestión de la propia identidad, se quiere evitar que nuestra vida digital sea la fuente de ingresos de las empresas que recopilan y comercializan nuestros datos, gustos, deseos, opiniones \dots

Para registrar todas operaciones que requieren validar nuestros datos de identidad, entrega de credenciales al usuario, presentación de credenciales a los proveedores de servicios y retirada de derecho de uso de la credencial, es necesaria una \textit{Blockchain} (cadena de bloques) o una DLT (Distributed Ledger Technology – Tecnología de registros distribuidos).

\section{Descentralización}
Las redes \textit{p2p} \cite{web:p2p} y la \textit{blockchain} \cite{web:blockchain} son ordenadores conectados ente si. Estos desbloquean las puertas de la descentralización, ya que permiten comunicar a personas a lo largo del mundo de manera segura y sin necesidad de infraestructuras caras ni grandes servidores.

Este proyecto introduce tanto alojamiento como transporte de información distribuida con computación deslocalizada. Esto significa que tanto los datos como el procesamiento queda extendido por toda la red. Gracias a esto, podemos traer la información y el procesamiento más cerca del usuario.

\section{Motivación}
\begin{itemize}
    \item Este proyecto ha sido elegido por el \textbf{aumento de robos de identidad}. El robo de identidad es el delito de más rápido crecimiento en el mundo. Se hace para acceder a ciertos recursos o para la obtención de créditos y otros beneficios en nombre de esa persona o para perjudicarla, con el fin de calumniar o desacreditar su vida profesional o familiar, o sembrar dudas sobre su salud, etc. \cite{web:robo_de_identidad}.
    \item Otro motivo, es el \textbf{incremento de los ciberataques a servidores}. Durante el 2021, hubo en España una media diaria de 40.000 ciberataques, lo que supuso un incremento del 125 por ciento con respecto al año anterior. Los más impactantes fueron: Servicio Público de Empleo Estatal (SEPE), por el \textit{ransomware} Ryuk, el 9 de marzo: su web quedó paralizada durante días, retrasando centenares de miles de citas; The Phone House, el 11 de abril, mediante \textit{ransomware}: robaron información de más de 3 millones de clientes; Glovo, el 29 de abril: acceso no autorizado a sus sistemas a través de una antigua interfaz del panel de administración: el 'hacker' puso a la venta en Internet el acceso a los datos de las cuentas de clientes y repartidores; MediaMarkt, a principios del mes de noviembre, en la preparación de la campaña de Black Friday, por \textit{ransomware}: hubo más de 3.000 sistemas Windows afectados, bloqueó todas las gestiones de la web \cite{web:ciberataques }.
    \item Además, por la \textbf{manipulación de procesos electorales y opinión pública}. El caso más reciente e importante fue el de Facebook-Cambridge Analytica: en la década del 2010, la consultora británica Cambridge Analytica recopiló datos de millones de usuarios de Facebook sin su consentimiento, principalmente para utilizarlos para propaganda política. Participó en la campaña presidencial de Trump y en la campaña a favor del Brexit. En julio de 2019 la Comisión Federal de Comercio (FTC) impuso a Facebook una multa de 5000 millones de dólares por violaciones de la privacidad de 87 millones de usuarios. En octubre de 2019, Facebook aceptó pagar una multa de 500 000 libras en el Reino Unido por exponer los datos de sus usuarios \cite{web:CA}.
    \item También por la \textbf{comercialización y los negocios montados con el análisis de datos rastreados}. La huella digital en internet o sombra digital es el conjunto único de actividades, acciones, contribuciones y comunicaciones digitales rastreables, que se manifiestan en Internet o en dispositivos digitales. Estos datos determinan a qué ha reaccionado el usuario o cómo han sido influenciados. El controlador de los datos puede determinar cómo y por qué compran y cómo se comportan las personas \cite{web:huella}. Los Gigantes tecnológicos (Big Tech), son las mayores empresas en tecnología de la información de la industria en los Estados Unidos \cite{web:gigantes}.
    \item Por la \textbf{línea marcada por la Unión Europea}. En 2018, 27 Estados miembros (entre ellos España), Liechtenstein y Noruega crearon la European Blockchain Partnership (EBP). El objetivo de este grupo es el desarrollo de una infraestructura Europea de Servicios de Blockchain o EBSI. La red EBSI deberá hacer realidad el intercambio de datos entre países de una manera sencilla y supondrá un mejor acceso a los servicios transeuropeos. EBSI ha sido diseñada bajo cinco principios: pública y permisionada, escalable, abierta, sostenible e interoperable. La versión 1 de EBSI se estuvo probando durante 2020 y España instaló tres nodos que mantiene operativos. España ha participado sobre todo en “Identidad Digital Soberana Europea (European Self-Sovereign Identity – ESSIF)”, que permite a los ciudadanos europeos gestionar sus datos de identidad y también en el de uso de “Diplomas” mediante el cual un ciudadano podrá presentar la credencial verificable de sus titulaciones en cualquier Universidad Europea o cualquier empresa que se lo solicite. Este año se ha puesto en marcha la versión 2 (ESBI v.2), con el objetivo de impulsar y extender el caso de uso de Diplomas y el pasaporte de Seguridad Social \cite{web:EBSI}. En junio de 2021 se publicó la propuesta de la Comisión Europea para ampliar el Sistema Europeo de Reconocimiento de Identidades Electrónicas (eIDAS), un conjunto de normas para la identificación electrónica y los servicios de confianza para transacciones electrónicas en el mercado único europeo \cite{web:eIDAS}. La versión 2 pretende proporcionar, para la utilización transfronteriza:
    \begin{itemize}
        \item Acceso a soluciones de identidad electrónica altamente seguras y fiables.
        \item La garantía de que los servicios públicos y privados puedan apoyarse en soluciones de identidad digital fiables y seguras.
        \item La garantía de que las personas físicas y jurídicas puedan utilizar soluciones de identidad digital.
        \item La seguridad de que dichas soluciones presenten un conjunto de atributos y permitan el intercambio selectivo de datos de identidad, y de que dichos datos se limiten a las necesidades del servicio específico solicitado.
        \item La garantía de la aceptación de los servicios de confianza en la Unión Europea (UE) y de la igualdad de condiciones para su prestación \cite{web:CE}.
    \end{itemize}

\end{itemize}

\newpage
\thispagestyle{empty}
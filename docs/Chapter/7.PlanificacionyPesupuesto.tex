\chapter{Planificación y presupuesto}\label{Planificación y presupuesto}
%Función que crea el título de capítulo y al cual se le da el nombre deseado a través de su parámetro obligatorio. Al no tener la función el “*” se escribirá también en el título del documento las palabras “Capítulo 1: …”. Además se indica, mediante la función “\label”, la correspondiente etiqueta que lleva asociada. La etiqueta sirve para que en caso de que luego se quiera hacer referencia al capítulo se haga llamando etiqueta tal que se escribiría “La información correspondiente a dicho tema se encuentra en el capítulo \ref{Int}.”

\thispagestyle{fancy}
%Función que determina que durante este capítulo se aplique el estilo Fancy.

\fancyhead[LE]{\thechapter.Planificación y presupuesto} 
%Función que se utiliza para indicar que en las páginas impares, aparezca en el encabezado en la parte izquierda, el número del capítulo con su correspondiente nombre.
El presupuesto de este proyecto se ha dividido en dos apartados claves: recursos humanos y recursos materiales.
\section{Recursos humanos}
Para poder tener un control de las horas dedicadas a este proyecto, se ha usado el tiempo dedicado más sobrecostes añadidos por semanas de trabajo extras. El proyecto originalmente estaba diseñado para dedicar 300 horas de trabajo a lo largo de un cuatrimestre de duración. El trabajo no se inició al inicio del cuatrimestre. Para empezar a computar las 300 horas de trabajo se necesitaba generar los objetivos y acotar todos los requisitos. Esta fase inicial ha tenido una duración de 100 horas; estas horas, se dedicaron a investigar el estado del arte alrededor del SSI y las tecnologías distribuidas. Para la creación de la memoria y la documentación del proyecto se utilizaron 100 horas más. Las 300 horas se dedicaron a analizar, diseñar, desarrollar y validar el proyecto. Desde, elegir las librerías y la blockchain a utilizar, hasta escribir todo el código necesario para generar una demo funcional. En total se computaron 500 horas de trabajo y utilizando un calculo de pago de 25\euro/hora, nos da un total de \textbf{12500\euro}.
\section{Recursos materiales}
Para el desarrollo de este proyecto, hay que contar con los materiales utilizados. Se ha usado un \textit{hardware} potente por su necesidad de procesamiento y RAM. En concreto hemos utilizado un portátil "Huawei matebook x pro 2017". El coste de este dispositivo es de \textbf{1000\euro}.
\section{Coste total}
\begin{table}[H]
    \centering
    \begin{tabular}{|l|c|}
        \hline
        \textbf{Recursos humanos}       & 12500\euro    \\
        \hline
        \textbf{Recursos materiales}    & 1000\euro     \\
        \hline
        \textbf{Total}                  & 13500\euro    \\
        \hline
    \end{tabular}
    \caption{Tabla con el desglose de costes del proyecto}
    \label{fg:dinero}
\end{table}
En esta tabla \ref{fg:dinero} se especifica todo el coste de planificación y desarrollo del proyecto.
\newpage
\thispagestyle{empty}
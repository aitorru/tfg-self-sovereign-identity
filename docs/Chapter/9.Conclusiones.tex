\chapter{Conclusiones}\label{Conclusiones}
%Función que crea el título de capítulo y al cual se le da el nombre deseado a través de su parámetro obligatorio. Al no tener la función el “*” se escribirá también en el título del documento las palabras “Capítulo 1: …”. Además se indica, mediante la función “\label”, la correspondiente etiqueta que lleva asociada. La etiqueta sirve para que en caso de que luego se quiera hacer referencia al capítulo se haga llamando etiqueta tal que se escribiría “La información correspondiente a dicho tema se encuentra en el capítulo \ref{Int}.”

\thispagestyle{fancy}
%Función que determina que durante este capítulo se aplique el estilo Fancy.

\fancyhead[LE]{\thechapter.Conclusiones}
%Función que se utiliza para indicar que en las páginas impares, aparezca en el encabezado en la parte izquierda, el número del capítulo con su correspondiente nombre.
A lo largo de este proyecto se han presentado multitud de problemas relacionados con las identidades y los datos. Así mismo, también se han propuesto múltiples soluciones que pueden llegar a convertirse en el internet del mañana. Esta combinación de tecnologías no es nada más que un intento de predecir el futuro. Se dislumbra un futuro prometedor, pero por ahora solo es poco más que un concepto. Estas tecnologías no están preparadas para ser utilizadas en \textit{producción}. Tanto \verb|IPFS| como \verb|js-ipfs| están en \textit{alpha}. \verb|OrbitDB|, pilar central para coordinar la información, dice lo siguiente:
\begin{quote}
    NOTE! OrbitDB is alpha-stage software. It means OrbitDB hasn't been security audited and programming APIs and data formats can still change. We encourage you to reach out to the maintainers if you plan to use OrbitDB in mission critical systems. \cite{web:OrbitDBalpha}
\end{quote}
Es posible que el futuro no consiga ser totalmente descentralizado, pero por lo menos pueda llegar a ser lo más \textit{local-first} \cite{web:localfirst} posible. En definitiva un internet que tenga al usuario en centro, sin muros que dificulten la entrada y, sobre todo, sin cámaras.
\newpage
\thispagestyle{empty}
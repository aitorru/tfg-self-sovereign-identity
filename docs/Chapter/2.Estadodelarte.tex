\chapter{Estado del arte}\label{EdA}
%Función que crea el título de capítulo y al cual se le da el nombre deseado a través de su parámetro obligatorio. Al no tener la función el “*” se escribirá también en el título del documento las palabras “Capítulo 1: …”. Además se indica, mediante la función “\label”, la correspondiente etiqueta que lleva asociada. La etiqueta sirve para que en caso de que luego se quiera hacer referencia al capítulo se haga llamando etiqueta tal que se escribiría “La información correspondiente a dicho tema se encuentra en el capítulo \ref{Int}.”

\thispagestyle{fancy}
%Función que determina que durante este capítulo se aplique el estilo Fancy.

\fancyhead[LE]{\thechapter.Estado del arte} 

\section*{SSI}
SSI (Self sovereign identity) en ingles, es un un concepto en el la propia persona por existir, se representa a si mismo en internet. Esto, quiere decir que en internet nos dejaríamos de identificar utilizando correos electrónicos y pasaríamos a identificarnos por nuestra propia voluntad y una unión de clave publica y clave privada.

\section*{Introducción a las blockchain}
La blockchain, es un elemento importante para este proyecto. Su implementación, permite una comunicación segura y anónima entre personas, sin necesidad de ser verificada por terceros. Las blockchain, vienen en muchas formas y tipos, algunas siendo descentralizadas. Las mas populares, funcionan de manera puramente descentralizada, usando un sistema de prueba de trabajo para verificar todas las transacciones.

\newpage
\thispagestyle{empty}